% Introduction
%%%%%%%%%%%%%%%%%%%%%%%%%%%%%%%%%%%%%%%%%%%%%

% FRENCH
%%%%%%%%%%%%%%%%%%%%%%%%%%%%%%%%%%%%%%%%%%%%%
Dans une Finlande où le vélo, la marche à pieds et les transports en communs sont de plus en plus utilisés, il devient nécessaire de se munir de moyens d'informations les plus utiles possibles afin de pouvoir anticiper les déplacements.

A l’ère actuelle, les entreprises sont toujours à l’affût de nouveautés, notamment dans le domaine SIG, que ce soit question de nouveaux outils ou de technologie. Chaque entreprises développent et commercialisent leurs outils d'information traffic. C'est dans une optique de partage de l'information, et du libre accès que va s'axer notre projet.

Aujourd'hui, open source et open data sont des sources d'information de plus en plus utilisées et permettent d'évoluer en communauté coopérative. Notre objectif principal, pour la réalisation de ce projet, est de récupérer les données et les outils open sources sur lesquels n'importe qui pourrait s'appuyer pour développer un modèle de consultation de données sur les habitudes de transport d'une ville. Notre cadre de recherche s'axera sur la ville de Hyvinkää et des données de déplacement à vélo.

Le travail réalisé se découpe en trois grandes parties : la première consiste à déterminer le modèle de transport théorique et mathématique des données qui seront consommées ; la seconde explique la façon dont une application peut être développée pour la consultation des données ; et enfin la troisième, et dernière partie, permet de combiner les précédentes parties afin d'obtenir un résultat consultable sur différentes plateformes.
