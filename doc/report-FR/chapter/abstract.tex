% ABSTRACT
%%%%%%%%%%%%%%%%%%%%%%%%%%%%%%%%%%%%%%%%%%%%%

% FRENCH
%%%%%%%%%%%%%%%%%%%%%%%%%%%%%%%%%%%%%%%%%%%%%

Les Systèmes d'Information Géographique (SIG) permettent, entre autres, d'acquérir, de traiter, d'organiser et de présenter des données géographiques, produisant ainsi des plans clairs, précis et intuitifs, et ce, à travers une composante web accessible depuis n'importe quel navigateur.

Le projet TraMap, proposé par HAMK et supervisé par Ramboll, est un outil simple et flexible permettant à un utilisateur de consulter des informations d'habitudes de transport issues de données géographiques en open source et d'en obtenir les informations sous forme d'application web et mobile.

C'est dans le cadre de ce besoin d'outil de consultation et d'exploitation des données que s'inscrit notre projet. Notre rôle est de rechercher des données open sources disponibles sur internet, d'en extraire les informations pour contruire un modèle d'habitude de transport, puis de développer une application de consultation des données multiplateformes. \\ \\

KEYWORDS
\par
\smallskip
\noindent \textbf{Keywords:} SIG, Modèle de Transport, Web, Algorithme, Sources Libres
